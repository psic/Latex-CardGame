%   COMMANDS ZUM ZUSAMMENBAUEN DER KARTEN
%   ---------------------------------------

%   TikZ/PGF Settings für die Karten
\pgfmathsetmacro{\cardwidth}{6}
\pgfmathsetmacro{\cardheight}{9}
\pgfmathsetmacro{\imagewidth}{\cardwidth}
% \pgfmathsetmacro{\imageheight}{0.75*\cardheight}
\pgfmathsetmacro{\imageheight}{0.77*\cardheight}
\pgfmathsetmacro{\stripwidth}{0.7}
\pgfmathsetmacro{\strippadding}{0.2}
\pgfmathsetmacro{\textpadding}{0.1}
\pgfmathsetmacro{\titley}{\cardheight-\strippadding-1.5*\textpadding-0.5*\stripwidth}


%   Formen der einzelnen Kartenelemente/-bestandteile
\def\shapeCard{(0,0) rectangle (\cardwidth,\cardheight)}
\def\shapeLeftStripLong{(\strippadding,-0.2) rectangle (\strippadding+\stripwidth,\cardheight-\strippadding-\strippadding-1)}
\def\shapeLeftStripShort{(\strippadding,\cardheight-\strippadding-1) rectangle (\strippadding+\stripwidth,\cardheight+0.2)}
\def\shapeRightStripShort{(\cardwidth-\stripwidth-\strippadding,\cardheight-\strippadding-1) rectangle (\cardwidth-\strippadding,\cardheight+0.2)}
\def\shapeTitleArea{(0,\cardheight-\strippadding) rectangle (\cardwidth,\cardheight-2*\stripwidth)}
\def\shapeContentArea{(2*\strippadding+\stripwidth,0.1*\cardheight) rectangle (\cardwidth+0.2,-0.2)}


%   Stylings für die Elemente definieren
\tikzset{
    %   runde Ecken für die Karten
    cardcorners/.style={
        rounded corners=0.2cm
    },
    %   runde Ecken für die "Fähnchen"
    elementcorners/.style={
        rounded corners=0.1cm
    },
    %   Schlagschatten für die "Fähnchen"
    stripshadow/.style={
        drop shadow={
            opacity=.5,
            shadow,
            color=black
        }
    },
    %   Style für die "Fähnchen"
    strip/.style={
        elementcorners,
        stripshadow
    },
    %   Bild für das Kartenmotiv
    cardimage/.style={
        path picture={
            \node[below=-1.5mm] at (0.5*\cardwidth,\cardheight) {
%                \includegraphics[width=\imagewidth cm]{#1}
                 \includegraphics[height=\imageheight cm]{#1}
            };
        }
    },
}

\newcommand*{\myfont}{\fontfamily{Josefinsans}\selectfont}

%   TikZ-Raster
\newcommand{\carddebug}{
    \draw [step=1,help lines] (0,0) grid (\cardwidth,\cardheight);
}

%   Rahmen der Karte
\newcommand{\cardborder}{
    \draw[lightgray,cardcorners] \shapeCard;
}

%   Hintergrund der Karte
\newcommand{\cardbackground}[1]{
    \draw[cardcorners, cardimage=#1] \shapeCard;
}

%   Kategorie der Karte
\newcommand{\cardtype}[3]{
    %   First we fill the intersecting area
    %   The \clip command does not allow options, therefore 
    %   we have to use a scope to set the even odd rule.
    \begin{scope}[even odd rule]
        %   Define a clipping path. All paths outside shapeCard will
        %   be cut because the even odd rule is set.
        \clip[cardcorners] \shapeCard;
        % Fill shapeLeftStripLong and shapeLeftStripShort.
        %   Since the even odd rule is set, only the card will be filled.
        \fill[strip,#1] \shapeLeftStripLong node[rotate=90,above left=0.9mm,font=\normalsize] {
            \color{white}\uppercase{#2}
        };
        \fill[strip,#1] \shapeLeftStripShort;
    \end{scope}

    \node at (\strippadding+\stripwidth-0.28,\cardheight-\strippadding-\strippadding-0.37) {\color{white}#3};
}
\newcommand{\cardtypeCharacter}{\cardtype{characterbg}{Charaktereigenschaft}{\hspace{-1mm}\LARGE\lefthand}}
\newcommand{\cardtypeAbility}{\cardtype{abilitybg}{Fähigkeit}{\hspace{-1mm}\Large\floweroneright}}
\newcommand{\cardtypeItem}{\cardtype{itembg}{Gegenstand}{\hspace{-1mm}\LARGE\bomb}}
\newcommand{\cardtypeTest}{\cardtype{testbg}{Testkarte}{\hspace{-1.4mm}\huge\ding{78}}}

%   Titel der Karte
\newcommand{\cardtitle}[1]{
    %\draw[pattern=soft crosshatch,rounded corners=0.1cm] \shapeTitleArea;
    \fill[elementcorners,titlebg,opacity=.50] \shapeTitleArea;
    \node[text width=\cardwidth cm] at (0.5*\cardwidth,\titley) {
        \begin{center}
            \color{white}\uppercase{\normalsize\textbf #1}
        \end{center}
    };
}

%   Inhalt der Karte
\newcommand{\cardcontent}[3]{
    \begin{scope}[even odd rule]
        \clip[cardcorners] \shapeCard;
        \fill[elementcorners,contentbg] \shapeContentArea;
    \end{scope}
    
    %Liseret blanc
    \begin{scope}[shift={(0.25*\cardwidth,0.15*\cardheight)},scale=0.4]
    \fill [white](0,0) circle (0.2+0.62*\cardwidth);
    \end{scope}
    \begin{scope}[shift={(0.75*\cardwidth,0.15*\cardheight)},scale=0.4]
    \fill [white](0,0) circle (0.2+0.62*\cardwidth);
    \end{scope}
    \begin{scope}[shift={(0.5*\cardwidth,0.254*\cardheight)},scale=0.25]
        \fill [white](0,0) circle (0.3+0.68*\cardwidth);
    \end{scope}
    
    
    % Black filler
    \begin{scope}
        \fill[cardcorners,white](0.0,0.0) rectangle(\cardwidth,0.25*\cardheight);
       \fill[cardcorners,black!10](0.05,0.05) rectangle(\cardwidth-0.05,0.25*\cardheight-0.05);
    \end{scope}
    \begin{scope}[shift={(0.25*\cardwidth,0.15*\cardheight)},scale=0.4]
%     \fill [black!10](0,0) circle (0.62*\cardwidth);
    \fill[black!10] (0,0) -- (60:0.62*\cardwidth) arc (60:150:0.62*\cardwidth) -- cycle;
    \end{scope}
    \begin{scope}[shift={(0.75*\cardwidth,0.15*\cardheight)},scale=0.4]
%     \fill [black!10](0,0) circle (0.62*\cardwidth);
    \fill[black!10] (0,0) -- (120:0.62*\cardwidth) arc (120:30:0.62*\cardwidth) -- cycle;
    \end{scope}
    \begin{scope}[shift={(0.5*\cardwidth,0.254*\cardheight)},scale=0.25]
        \fill [black!10](0,0) circle (0.68*\cardwidth);
    \end{scope}
%     \begin{scope}[shift={(0.75*\cardwidth,0.15*\cardheight)},scale=0.4]
%     \fill [black!10](0,0) circle (0.62*\cardwidth);
% %      \fill[red] (0,0) -- (120:0.62*\cardwidth) arc (120:30:0.62*\cardwidth) -- cycle;
% 
%     \end{scope}
    
    \begin{scope}[shift={(0.5*\cardwidth,0.254*\cardheight)},scale=0.25]
    \ifthenelse{#3 > 3}
    {\draw[draw=white,fill=green,thick] (0:2) -- (0:3.5) arc(0:45:3.5) -- (45:2) arc(45:0:2) -- cycle }
    {\draw[draw=white,fill=black,thick] (0:2) -- (0:3.5) arc(0:45:3.5) -- (45:2) arc(45:0:2) -- cycle };
    \ifthenelse{#3 > 2}
    {\draw[draw=white,fill=green!70,thick] (90:2cm)-- (90:3.5cm) arc (90:45:3.5) -- (45:2cm) arc (45:90:2)}
    {\draw[draw=white,fill=black!70,thick] (90:2cm)-- (90:3.5cm) arc (90:45:3.5) -- (45:2cm) arc (45:90:2)};
    \ifthenelse{#3 > 1}
    {\draw[draw=white,fill=green!50,thick] (135:2cm)-- (135:3.5cm) arc (135:90:3.5) -- (90:2cm) arc (90:135:2)}
    {\draw[draw=white,fill=black!50,thick] (135:2cm)-- (135:3.5cm) arc (135:90:3.5) -- (90:2cm) arc (90:135:2)};

    \draw[draw=white,fill=green!30,thick] (135:2cm)-- (135:3.5cm) arc (135:180:3.5) -- (180:2cm) arc (180:135:2);
     \fill[elementcorners,white](-1.51,-1.07) rectangle(1.45,0.89);
    \fill[elementcorners,black](-1.33,-0.89) rectangle(1.27,0.71);
    \draw (0,-0.12) node[white]{#3/4};


    \end{scope}
    
    \begin{scope}[shift={(0.25*\cardwidth,0.15*\cardheight)},scale=0.35]
        \fill [white](0,0) circle (3.9cm);
        \fill [red!50](0,0) circle (3.7cm);
        \draw (0,0.8) node[white]{\footnotesize KM/H};
        \draw (0,-0.7) node[white]{\footnotesize Vitesse};
%         \fill[black!70,rotate=145] (0,0.2)--(0:2.75) -- (0,-0.2)--cycle;
         \fill[black!70,rotate=330-#1] (0,0.2)--(0:2.75) -- (0,-0.2)--cycle;

        \fill [black!70](0,0) circle (0.4cm);
        \fill[white,rotate=330-#1] (0,0.1)--(0:1.5) -- (0,-0.1)--cycle;
        \fill [white](0,0) circle (0.2cm);
        \fill [red](0,0) circle (0.1cm);
        \draw[red,rotate=330-#1] (0,0)--(0:0.5);

        
        \foreach \a in {0, 2,...,300}
            {\draw[draw=black] (\a+330:3.50) -- (\a+330:3.70); }
        
        
        \foreach \a in {0, 10,...,300}
            {\draw[draw=black] (\a+330:3.30) -- (\a+330:3.70); }
        
        \foreach \a in {0, 20,...,300}
            {\draw[draw=black] (\a+330:3) -- (\a+330:3.70); }
        
            
         \foreach \a in {0,20,..., 300}
                {\pgfmathtruncatemacro\result{(360 - \a)}
                \draw (\a +330: 2.5) node[white, rotate=0, scale=0.7] {\ttfamily\tiny\result};}
        
%         \fill[elementcorners,white](-1.1,-5) rectangle(1.1,-3.7);
%         \fill[elementcorners,black](-1,-4.9) rectangle(1,-3.8);
%         \draw (0,-4.35) node[white]{#1};

         \fill[elementcorners,white](0.27,-3.17) rectangle(2.51,-1.83);
        \fill[elementcorners,black](0.39,-3.05) rectangle(2.39,-1.95);
        \draw (1.39,-2.5) node[white]{#1};

                
    \end{scope}

    \begin{scope}[shift={(0.75*\cardwidth,0.15*\cardheight)},scale=0.35]
        \fill [white](0,0) circle (3.9cm);
        \fill [blue!50](0,0) circle (3.7cm);
        \draw (0,0.8) node[white]{\footnotesize Chevaux};
        \draw (0,-0.7) node[white]{\footnotesize Puissance};
        \pgfmathtruncatemacro\result{((1500-#2)/5)-30)}
        \fill[black!70,rotate=\result] (0,0.2)--(0:2.75) -- (0,-0.2)--cycle;
       

        % 330 (-30) - 1500
        % 0         - 1350
        % 90        - 900
        % 180       - 450
        % 630 (270) - 0
        
        \fill [black!70](0,0) circle (0.4cm);
        \fill[white,rotate=\result] (0,0.1)--(0:1.5) -- (0,-0.1)--cycle;
        \fill [white](0,0) circle (0.2cm);
        \fill [red](0,0) circle (0.1cm);
        \draw[red,rotate=\result] (0,0)--(0:0.5);

%         \foreach \a in {0, 2,...,300}
%             {\draw[draw=black] (\a+330:3.50) -- (\a+330:3.70); }
%         
        
        \foreach \a in {0, 10,...,300}
            {\draw[draw=black] (\a+330:3.30) -- (\a+330:3.70); }
        
        \foreach \a in {0, 20,...,300}
            {\draw[draw=black] (\a+330:3) -- (\a+330:3.70); }
        
            
         \foreach \a in {0,20,..., 300}
                {\pgfmathtruncatemacro\result{(1500 -(5 * \a))}
                \draw (\a +330: 2.5) node[white, rotate=0, scale=0.7] {\ttfamily\tiny\result};}
        
%         \fill[elementcorners,white](-1.1,-5) rectangle(1.1,-3.7);
%         \fill[elementcorners,black](-1,-4.9) rectangle(1,-3.8);
%         \draw (0,-4.35) node[white]{#2};
        
        \fill[elementcorners,white](0.27,-3.17) rectangle(2.51,-1.83);
        \fill[elementcorners,black](0.39,-3.05) rectangle(2.39,-1.95);
        \draw (1.39,-2.5) node[white]{#2};
                
    \end{scope}
%     \node[below right, text width=(\cardwidth-2*\strippadding-\stripwidth-2*\textpadding-0.3)*1cm] at (2*\strippadding+\stripwidth+\textpadding,0.5*\cardheight-\textpadding) {
%         \textit{\normalsize #1 km}
%         };
%         
%         
%         
%     
%     \node[below right, text width=(\cardwidth-2*\strippadding-\stripwidth-2*\textpadding-0.3)*1cm] at (2*\strippadding+\stripwidth+\textpadding,3) {
% %         \vrule width \textwidth height 2pt \\[-2pt]
%         \vspace{0.2cm}
%         \faIcon{youtube} {\scriptsize #2 chevaux}
%     };
%     
%     \node[below right, text width=(\cardwidth-2*\strippadding-\stripwidth-2*\textpadding-0.3)*1cm] at (2*\strippadding+\stripwidth+\textpadding,5) {
% %         \vrule width \textwidth height 2pt \\[-2pt]
%         \vspace{0.2cm}
%         \faIcon{youtube} {\scriptsize #3 /4}
%     };
}

%   Preis der Karte
\newcommand{\cardprice}[1]{
    \begin{scope}[even odd rule]
        \clip[cardcorners] \shapeCard;
        \fill[strip,pricebg] \shapeRightStripShort;
    \end{scope}
    \node at (\cardwidth-0.5*\stripwidth-\strippadding,\titley-0.1) {\color{black}#1};
}
